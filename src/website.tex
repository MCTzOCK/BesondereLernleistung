\documentclass[main.tex]{subfiles}

\begin{document}
    \section{Konzept}
    Die Webseite von CodeUp ist eine Plattform für Menschen, die sich für das Programmieren interessieren und ihre Fähigkeiten verbessern möchten.
    Der Hauptfokus liegt hierbei auf der Entwicklung einer ganzheitlichen Lernumgebung, die über das reine Lernen von Programmiersprachen und Konzepten hinausgeht und alle wichtigen Aspekte der Software-Entiwcklung abdeckt.
    Dazu gehören unter anderem die Planung und Umsetzung von Projekten, die Zusammenarbeit mit anderen Entwicklern und die Veröffentlichung von eigenen Projekten.
    Da sich CodeUp vor allem an Schüler richtet, die noch keine oder nur wenig Erfahrung im Programmieren haben, wurden vor allem Unterrichtsrelevante Themen in den Vordergrund gestellt.
    Ebenfalls gibt es Integrationsmöglichkeiten für die Verwendung in der Schule oder in anderen Bildungsinstitutionen.
    Die Webseite ist so aufgebaut, dass sie sowohl für Anfänger als auch für Fortgeschrittene geeignet ist und eine Vielzahl von Funktionen bietet, die für beide Zielgruppen relevant sind.
    Auch die Wahl des Lernmediums ist wichtig, weshalb nach intensiver Recherche entschieden wurde, dass das Lernen über interaktive (Video-)Kurse am effektivsten ist.
    Diese interaktiven Kurse sind in verschiedene Lektion unterteilt, die jeweils exakt einen Themenaspekt abdecken.


    \section{Technologie-Stack}
    CodeUp wurde mit modernen Web-Technologien entwickelt, um eine hohe Benutzerfreundlichkeit und Funktionalität zu gewährleisten.
    Da die Webseite aus vielen verschiedenen Komponenten besteht, wurde sie in einer Monorepo-Struktur organisiert.
    Für diese Monorepo-Struktur wurde das Werkzeug \dq Turborepo\dq\footnote{\bscite{turborepo}}\ verwendet, welches eine einfache Verwaltung von mehreren Paketen ermöglicht.
    Die Wahl der Programmiersprache fiel auf TypeScript, da diese eine statische Typisierung ermöglicht und so die Code-Qualität erhöht und die Fehleranfälligkeit reduziert.
    Im Kern besteht CodeUp aus einem Frontend- und einem Backend-Teil, die durch die Bibliothek \dq Next.js\dq\footnote{\bscite{nextjs}}\ miteinander verbunden sind.
    Allerdings wird nicht der von Next.js bereitgestellte HTTP-Server verwendet, sondern ein eigener, der auf dem Node.js-Modul \dq http\dq\ basiert.
    Für die Kommunikation zwischen Frontend und Backend wird eine REST-API verwendet, die über eine eigene Bibliothek abstrahiert wird (siehe~\ref{subsec:web_srv_communication}).
    Die Daten werden in einer MongoDB-Datenbank gespeichert, während Dateien in einem S3-kompatiblen Object-Storage abgelegt werden.
    Für die Benutzeroberfläche fiel die Wahl auf das Design-System \dq Chakra UI\dq\footnote{\bscite{chakraui}}, das eine Vielzahl an vorgefertigten Komponenten bietet und eine einfache Anpassung ermöglicht.
    Chakra UI wurde für CodeUp angepasst und um ein eigenes Theme erweitert, welches Farben, Schriftarten und andere Design-Elemente enthält.
    Die Webseite ist responsive gestaltet, sodass sie auf allen Geräten optimal dargestellt wird, was allerdings nicht in allen Bereichen (z.B.~Code-Editoren) technisch möglich ist.
    Durch die Oranisation in einer Git-Repository wird die Versionskontrolle gewährleistet und Fehler können frühzeitig erkannt und behoben werden.
    Notfalls kann auf ältere Versionen zurückgegriffen werden, um Fehler zu beheben.
    Bei der tatsächlichen Entwicklung wird auf JetBrains WebStorm\footnote{\bscite{webstorm}} als IDE gesetzt, da diese eine Vielzahl an Funktionen bietet und so produktives Arbeiten ermöglicht.


    \subsection{Sicherheit}
    Für die sichere Speicherung und spätere Validierung von Anmeldedaten wird das Password-Hashing-Verfahren sha256 verwendet.
    Dieses Verfahren bietet eine sogenannte Einwegverschlüsselung, sodass Klartext zwar in einen Hash umgewandelt werden kann, dieser aber nicht wieder in Klartext zurückverwandelt werden kann.
    Dies ist besonders wichtig, da die Sicherheit der Benutzerdaten oberste Priorität hat.
    Wenn ein Benutzer sich anmeldet, wird das eingegebene Passwort mit dem oben genannten Verfahren gehasht und mit dem in der Datenbank gespeicherten Hash verglichen.
    Sollten die beiden Hashes übereinstimmen, wird der Benutzer authentifiziert und erhält einen JWT (JSON Web Token)\footnote{\bscite{jwt}}, der für die weitere Kommunikation mit dem Server verwendet wird.
    Dieser JWT enthält Informationen über den Benutzer und wird bei jeder Anfrage an den Server mitgeschickt, um den Benutzer zu authentifizieren.
    Generell wird sämtlicher Datenverkehr über eine gesicherte HTTPS-Verbindung abgewickelt, um die Kommunikation zwischen Client und Server zu verschlüsseln.

    \subsection{Benutzeroberfläche und Design}
    Die Benutzeroberfläche wurde mit dem React.js\footnote{\bscite{react}} Framework entwickelt und besteht aus einer Vielzahl von Komponenten, die in einer Komponenten-Bibliothek organisiert sind.
    React ist eine Bibliothek, die es ermöglicht, Benutzeroberflächen aus wiederverwendbaren Komponenten zu erstellen und gleichzeitig eine interaktive Benutzererfahrung zu bieten.
    Um die Benutzerfreundlichkeit zu erhöhen, wurde die Webseite möglichst übersichtlich und intuitiv gestaltet.
    Das heißt konkret, dass alle wichtigen Funktionen leicht zugänglich sind und die Navigation so einfach wie möglich gestaltet wurde.
    Das Hauptmenü bildet hierbei den zentralen Anlaufpunkt für alle Funktionen und ist immer sichtbar, unabhängig von der Position auf der Webseite.
    Ebenfalls wurden mehrere Sprachen unterstützt, um die Webseite für eine größere Zielgruppe zugänglich zu machen.
    Diese Sprachen sind allerdings zum aktuellen Zeitpunkt nur für Text-basierte Inhalte verfügbar, da die Übersetzung von Videos und anderen dynamischen Inhalten zu aufwendig ist.
    Das Design der Webseite wurde so gewählt, dass es modern und ansprechend wirkt, aber gleichzeitig nicht zu überladen ist.
    Die Farbgebung ist dezent und aufeinander abgestimmt, um eine angenehme Benutzererfahrung zu gewährleisten.
    So wurde generell auf grelle Farben und zu viele Animationen verzichtet, um die Ablenkung des Benutzers zu minimieren.
    Die Farbkomposition besteht als dunklen Grautönen für den Hintergrund und nicht wichtige Elemente, während wichtige Elemente in einem hellen Gelb ({\color{cyellow}\texttt{\#F7DE1F}}) hervorgehoben werden.
    Es kommen allerdings selten auch andere Farben zum Einsatz, beispielsweise bei destruktiven Aktionen, die in einem kräftigen Rot ({\color{cred}\texttt{\#E03131}}) dargestellt werden.

    \subsection{Backend}
    Das Backend von CodeUp basiert auf dem Node.js-Framework und dem dazu gehörigen \dq http\dq-Modul, welches die Erstellung eines HTTP-Servers ermöglicht.
    Neben dem HTTP-Server wird auch ein WebSocket-Server verwendet, um eine bidirektionale Kommunikation zwischen Client und Server zu ermöglichen.
    Für die WebSocket-Kommunikation wird die Bibliothek \dq socket.io\dq \footnote{\bscite{socketio}}\ verwendet.
    Die Datenspeicherung und spätere Verarbeitung erfolgt über eine MongoDB-Datenbank.
    MongoDB wurde initial in Betracht gezogen, da es eine Nicht-Relationale Datenbank ist und somit eine hohe Flexibilität bietet, die für CodeUp zwingend erforderlich ist.
    In dieser MongoDB-Datenbank werden sowohl die Benutzerdaten, als auch sämtliche andere Daten der Platform gespeichert.
    Lediglich Dateien, die Benutzer hochladen oder durch das Anlegen von Projekte indirekt erstellen, werden separat im S3-kompatiblen Object-Storage MinIO gespeichert.
    MinIO ist eine Open-Source-Alternative zu Amazon S3 und ermöglicht die effiziente Speicherung von Dateien.

    \subsection{Kommunikation zwischen Frontend und Backend}\label{subsec:web_srv_communication}
    Die Kommunikation zwischen dem Frontend und dem Backend der Webseite erfolgt größtenteils über eine REST-API, also eine HTTP-Kommunikation zwischen dem Client und dem Server.
    Damit diese REST-API möglichst einfach in den Frontend-Code integriert werden kann, wurde eine separate Bibliothek entwickelt, die die Kommunikation abstrahiert und vereinfacht.
    Diese Bibliothek wird in allen Komponenten der Webseite verwendet und umfasst alle möglichen Funktionen und Endpunkte, die der Server bereitstellt.
    Benutzer können die CodeUp-REST-API auch in eigenen Projekten verwenden, weshalb die Bibliothek auch als eigenständiges Paket veröffentlicht wurde.
    Eine beispielhafte Nutzung der Bibliothek zur Authentifizierung eines Benutzers könnte wie folgt aussehen:
    \begin{lstlisting}[language=javascript]
import { REST } from "@codeupspace/rest";
REST.Account.loginWithUsername({
    username: "test",
    password: "test"
});
    \end{lstlisting}
    In diesem Beispiel wird ein Benutzer mit dem Benutzernamen \dq test\dq \ und dem Passwort \dq test\dq \ authentifiziert.
    Die REST-API-Bibliothek kümmert sich um die korrekte Kommunikation mit dem Server und gibt als Ergebnis immer ein Promise-Objekt des Typs \dq IResponse\dq\ zurück.
    Dies ist ein spezielles Interface, das aus einem Statuscode und einem Datenobjekt besteht:
    \begin{lstlisting}[language=javascript]
interface IResponse {
    status: number;
    payload: any;
}
    \end{lstlisting}
    Das Datenobjekt enthält die Antwort des Servers, die je nach Anfrage unterschiedlich sein kann.
    Der Statuscode gibt an, ob die Anfrage erfolgreich war oder nicht.

    \section{Funktionsumfang}
    \subsection{Generelle und globale Funktionen}
    \subsection{Lernen}
    \subsubsection{Kurse}
    CodeUp bietet eine Vielzahl an Kursen an, von denen jeder einzelne ein spezifisches Thema, wie beispielsweise eine Programmiersprache oder ein Konzept, abdeckt.
    Grundsätzlich besteht jeder Kurs aus mehreren Lektionen, die jeweils einen Teilaspekt des Themas behandeln.
    Es gibt zwei verschiedene Arten von Kursen:
    \begin{itemize}
        \item \textbf{Interaktive Kurse}: Diese Kurse sind Text-basiert und bieten interaktive Übungen an, die den Benutzer dazu auffordern, selber Code zu schreiben.
        Hierbei beginnt jede Lektion mit einem kurzen Text, der das Thema einführt, gefolgt von einer oder mehreren Übungen, die den Benutzer dazu auffordern, Code zu schreiben.
        Jede Übung ist von einem Beispiel begleitet, das dem Benutzer zeigt, wie die Lösung aussehen könnte.
        \item \textbf{Video-Kurse}: Diese Kurse bestehen aus einer Reihe von Videos.
        Hierbei gibt es nach jedem Video ein Quiz, das den Benutzer dazu ermutigt, sein Wissen zu testen und ggf. die Lektion zu wiederholen.
    \end{itemize}
    Um eine 
    \subsubsection{Challenges}
    \subsubsection{Zertifikate}
    \subsection{Austauschen}
    \subsubsection{Forum}
    \subsubsection{Discovery}
    \subsection{Planen}
    \subsubsection{Aufgaben-Planung}
    \subsubsection{Flußdiagramme}
    \subsection{Entwickeln}
    \subsubsection{IDE (v1)}
    \subsubsection{Editor (v2)}
    \subsubsection{Dev-Suite (v3)}
    \subsubsection{CodeUp-Kids}
    \subsubsection{Code-Snippets}
    \subsubsection{Projektideen}
\end{document}
