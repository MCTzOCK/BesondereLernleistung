\documentclass[main.tex]{subfiles}

\begin{document}

    \section{Technischer Hintergrund}
    Die Webseite von CodeUp und damit der Hauptteil des Projekts wurde mit modernen Web-Technologien entwickelt.
    Dabei verfügt sie sowohl über einen Frontend- als auch einen Backend-Teil, allerdings werden diese durch die Bibliothek \dq Next.js\dq \footnote{\bscite{nextjs}}\ miteinander vereint.
    Um einen besseren Überblick über die gesamte Code-Basis zu erhalten, ist CodeUp in einer Monorepo-Struktur organisiert.
    Das bedeutet, dass die Webseite in verschiedene Pakete, wie Komponenten für die Benutzeroberfläche, Backend-Logik und Interaktionslogik, unterteilt ist.
    Für die Umsetzung der Monorepo-Struktur wurde das Werkzeug \dq Turborepo\dq \footnote{\bscite{turborepo}}\ verwendet.
    \subsection{Frontend}

    \subsection{Backend}
    Das Backend von CodeUp basiert auf dem Node.js-Framework und dem dazu gehörigen \dq http\dq-Modul, welches die Erstellung eines HTTP-Servers ermöglicht.
    Neben dem HTTP-Server wird auch ein WebSocket-Server verwendet, um eine bidirektionale Kommunikation zwischen Client und Server zu ermöglichen.
    Für die WebSocket-Kommunikation wird die Bibliothek \dq socket.io\dq \footnote{\bscite{socketio}}\ verwendet.
    Die Datenspeicherung und spätere Verarbeitung erfolgt über eine MongoDB-Datenbank.
    MongoDB wurde initial in Betracht gezogen, da es eine Nicht-Relationale Datenbank ist und somit eine hohe Flexibilität bietet, die für CodeUp zwingend erforderlich ist.
    In dieser MongoDB-Datenbank werden sowohl die Benutzerdaten, als auch sämtliche andere Daten der Platform gespeichert.
    Lediglich Dateien, die Benutzer hochladen oder durch das Anlegen von Projekte indirekt erstellen, werden separat im S3-kompatiblen Object-Storage MinIO gespeichert.
    MinIO ist eine Open-Source-Alternative zu Amazon S3 und ermöglicht die effiziente Speicherung von Dateien.

    \subsection{Kommunikation zwischen Frontend und Backend}
    Die Kommunikation zwischen dem Frontend und dem Backend der Webseite erfolgt größtenteils über eine REST-API, also eine HTTP-Kommunikation zwischen dem Client und dem Server.
    Damit diese REST-API möglichst einfach in den Frontend-Code integriert werden kann, wurde eine separate Bibliothek entwickelt, die die Kommunikation abstrahiert und vereinfacht.
    Diese Bibliothek wird in allen Komponenten der Webseite verwendet und umfasst alle möglichen Funktionen und Endpunkte, die der Server bereitstellt.
    Benutzer können die CodeUp-REST-API auch in eigenen Projekten verwenden, weshalb die Bibliothek auch als eigenständiges Paket veröffentlicht wurde.
    Eine beispielhafte Nutzung der Bibliothek zur Authentifizierung eines Benutzers könnte wie folgt aussehen:
    \begin{lstlisting}[language=javascript]
        import { REST } from "@codeupspace/rest";
        REST.Account.login();
    \end{lstlisting}
\end{document}