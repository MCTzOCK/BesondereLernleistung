\documentclass[main.tex]{subfiles}

\begin{document}

    \section{Technischer Hintergrund}
    Die Webseite von CodeUp und damit der Hauptteil des Projekts wurde mit modernen Web-Technologien entwickelt.
    Dabei verfügt sie sowohl über einen Frontend- als auch einen Backend-Teil, allerdings werden diese durch die Bibliothek \dq Next.js\dq \footnote{\bscite{nextjs}}\ miteinander vereint.
    \subsection{Frontend}

    \subsection{Backend}
    Das Backend von CodeUp basiert auf dem Node.js-Framework und dem dazu gehörigen \dq http\dq-Modul, welches die Erstellung eines HTTP-Servers ermöglicht.
    Neben dem HTTP-Server wird auch ein WebSocket-Server verwendet, um eine bidirektionale Kommunikation zwischen Client und Server zu ermöglichen.
    Für die WebSocket-Kommunikation wird die Bibliothek \dq socket.io\dq \footnote{\bscite{socketio}}\ verwendet.
\end{document}