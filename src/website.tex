\documentclass[main.tex]{subfiles}

\begin{document}
    \section{Konzept}
    Die Webseite von CodeUp ist eine Plattform für Menschen, die sich für das Programmieren interessieren und ihre Fähigkeiten verbessern möchten.
    Der Hauptfokus liegt hierbei auf der Entwicklung einer ganzheitlichen Lernumgebung, die über das reine Lernen von Programmiersprachen und Konzepten hinausgeht und alle wichtigen Aspekte der Software-Entiwcklung abdeckt.
    Dazu gehören unter anderem die Planung und Umsetzung von Projekten, die Zusammenarbeit mit anderen Entwicklern und die Veröffentlichung von eigenen Projekten.
    Da sich CodeUp vor allem an Schüler richtet, die noch keine oder nur wenig Erfahrung im Programmieren haben, wurden vor allem Unterrichtsrelevante Themen in den Vordergrund gestellt.
    Ebenfalls gibt es Integrationsmöglichkeiten für die Verwendung in der Schule oder in anderen Bildungsinstitutionen.

    \section{Technologie-Stack}
    CodeUp wurde mit modernen Web-Technologien entwickelt, um eine hohe Benutzerfreundlichkeit und Funktionalität zu gewährleisten.
    Da die Webseite aus vielen verschiedenen Komponenten besteht, wurde sie in einer Monorepo-Struktur organisiert.
    Für diese Monorepo-Struktur wurde das Werkzeug \dq Turborepo\dq\footnote{\bscite{turborepo}}\ verwendet, welches eine einfache Verwaltung von mehreren Paketen ermöglicht.
    Die Wahl der Programmiersprache fiel auf TypeScript, da diese eine statische Typisierung ermöglicht und so die Code-Qualität erhöht und die Fehleranfälligkeit reduziert.
    Im Kern besteht CodeUp aus einem Frontend- und einem Backend-Teil, die durch die Bibliothek \dq Next.js\dq\footnote{\bscite{nextjs}}\ miteinander verbunden sind.
    Allerdings wird nicht der von Next.js bereitgestellte HTTP-Server verwendet, sondern ein eigener, der auf dem Node.js-Modul \dq http\dq\ basiert.
    Für die Kommunikation zwischen Frontend und Backend wird eine REST-API verwendet, die über eine eigene Bibliothek abstrahiert wird (siehe~\ref{subsec:web_srv_communication}).
    Die Daten werden in einer MongoDB-Datenbank gespeichert, während Dateien in einem S3-kompatiblen Object-Storage abgelegt werden.
    Für die Benutzeroberfläche fiel die Wahl auf das Design-System \dq Chakra UI\dq\footnote{\bscite{chakraui}}, das eine Vielzahl an vorgefertigten Komponenten bietet und eine einfache Anpassung ermöglicht.
    Chakra UI wurde für CodeUp angepasst und um ein eigenes Theme erweitert, welches Farben, Schriftarten und andere Design-Elemente enthält.
    Die Webseite ist responsive gestaltet, sodass sie auf allen Geräten optimal dargestellt wird, was allerdings nicht in allen Bereichen (z.B.~Code-Editoren) technisch möglich ist.
    Durch die Oranisation in einer Git-Repository wird die Versionskontrolle gewährleistet und Fehler können frühzeitig erkannt und behoben werden.
    Notfalls kann auf ältere Versionen zurückgegriffen werden, um Fehler zu beheben.

    \subsection{Frontend}
    Das Frontend von CodeUp ist mit Hilfe des React.js-Frameworks realisiert.
    Dieses ermöglicht die einfache Entwicklung von interaktiven Benutzeroberflächen.
    Beim Design der Oberfläche wurde besonderer Wert auf Funktionalität und Benutzerfreundlichkeit gelegt.
    Ebenfalls sollte die Webseite für Menschen mit einer Sehschwäche oder anderen Einschränkungen zugänglich sein, weshalb die Webseite barrierefrei gestaltet wurde.
    Um eine solche Barrierefreiheit zu gewährleisten, wurde die Benutzeroberflächen-Bibliothek \dq Chakra UI\dq \footnote{\bscite{chakraui}}\ verwendet.
    Diese Bibliothek bietet eine Vielzahl an vorgefertigten Komponenten an, die sich einfach in eine React-Anwendung integrieren lassen.
    Für Chakra-UI wurde allerdings ein eigenes Theme erstellt, um das Design an die Bedürfnisse von CodeUp anzupassen.
    Ein weiterer wichtiger Aspekt ist die responsive Gestaltung der CodeUp-Webseite, sodass sie auf allen Geräten, wie Desktops, Tablets und Smartphones, optimal dargestellt wird.
    Dies ist nicht an allen Punkten technisch möglich, da beispielsweise Code-Editoren auf kleinen Bildschirmen nur schwer zu bedienen sind.

    \subsection{Backend}
    Das Backend von CodeUp basiert auf dem Node.js-Framework und dem dazu gehörigen \dq http\dq-Modul, welches die Erstellung eines HTTP-Servers ermöglicht.
    Neben dem HTTP-Server wird auch ein WebSocket-Server verwendet, um eine bidirektionale Kommunikation zwischen Client und Server zu ermöglichen.
    Für die WebSocket-Kommunikation wird die Bibliothek \dq socket.io\dq \footnote{\bscite{socketio}}\ verwendet.
    Die Datenspeicherung und spätere Verarbeitung erfolgt über eine MongoDB-Datenbank.
    MongoDB wurde initial in Betracht gezogen, da es eine Nicht-Relationale Datenbank ist und somit eine hohe Flexibilität bietet, die für CodeUp zwingend erforderlich ist.
    In dieser MongoDB-Datenbank werden sowohl die Benutzerdaten, als auch sämtliche andere Daten der Platform gespeichert.
    Lediglich Dateien, die Benutzer hochladen oder durch das Anlegen von Projekte indirekt erstellen, werden separat im S3-kompatiblen Object-Storage MinIO gespeichert.
    MinIO ist eine Open-Source-Alternative zu Amazon S3 und ermöglicht die effiziente Speicherung von Dateien.

    \subsection{Kommunikation zwischen Frontend und Backend}\label{subsec:web_srv_communication}
    Die Kommunikation zwischen dem Frontend und dem Backend der Webseite erfolgt größtenteils über eine REST-API, also eine HTTP-Kommunikation zwischen dem Client und dem Server.
    Damit diese REST-API möglichst einfach in den Frontend-Code integriert werden kann, wurde eine separate Bibliothek entwickelt, die die Kommunikation abstrahiert und vereinfacht.
    Diese Bibliothek wird in allen Komponenten der Webseite verwendet und umfasst alle möglichen Funktionen und Endpunkte, die der Server bereitstellt.
    Benutzer können die CodeUp-REST-API auch in eigenen Projekten verwenden, weshalb die Bibliothek auch als eigenständiges Paket veröffentlicht wurde.
    Eine beispielhafte Nutzung der Bibliothek zur Authentifizierung eines Benutzers könnte wie folgt aussehen:
    \begin{lstlisting}[language=javascript]
import { REST } from "@codeupspace/rest";
REST.Account.loginWithUsername({
    username: "test",
    password: "test"
});
    \end{lstlisting}
    In diesem Beispiel wird ein Benutzer mit dem Benutzernamen \dq test\dq \ und dem Passwort \dq test\dq \ authentifiziert.
    Die REST-API-Bibliothek kümmert sich um die korrekte Kommunikation mit dem Server und gibt als Ergebnis immer ein Promise-Objekt des Typs \dq IResponse\dq\ zurück.
    Dies ist ein spezielles Interface, das aus einem Statuscode und einem Datenobjekt besteht:
    \begin{lstlisting}[language=javascript]
interface IResponse {
    status: number;
    payload: any;
}
    \end{lstlisting}
    Das Datenobjekt enthält die Antwort des Servers, die je nach Anfrage unterschiedlich sein kann.
    Der Statuscode gibt an, ob die Anfrage erfolgreich war oder nicht.
\end{document}
