\documentclass[main.tex]{subfiles}

\begin{document}
    \section{Konzept}
    Das Konzept der CodeUp-Mobile App ist es, die Funktionalitäten der Webseite auf mobilen Geräten verfügbar zu machen.
    Zwar ist die eigentliche Webseite bereits für mobile Geräte optimiert, jedoch kann mit einer nativen App ein noch besseres Erlebnis geschaffen werden.
    Die App soll dabei möglichst viele Funktionen der Webseite bereitstellen und somit den Nutzern ein nahtloses Erlebnis bieten.
    \section{Technologie-Stack}
    Die CodeUp-Mobile App wurde mit React Native\footnote{\bscite{react-native}} entwickelt.
    Weitere Informationen zu React Native finden sich in Anhang~\ref{sec:react-native}.
    Für die Benutzeroberfläche wurde die Bibliothek native-base\footnote{\bscite{native-base}} verwendet.
    Um die Entwicklung zu vereinfachen, wurde die App mit Expo\footnote{\bscite{expo}} entwickelt.
    Expo ist ein Framework, das die Entwicklung von React Native-Apps vereinfacht und beschleunigt.
    Für die Einbindung der Webseite wurde react-native-webview\footnote{\bscite{react-native-webview}} verwendet.
    \section{Funktionen}
    Die CodeUp-Mobile App ist, anders als die Webseite, nicht ohne Anmeldung nutzbar.
    Sollte ein Benutzer beim Start der App nicht angemeldet sein, wird er aufgefordert sich mit seinem CodeUp-Konto anzumelden.
    Hierbei kann er sich auch registrieren, allerdings ist eine Zurücksetzung des Passworts nur über die Webseite möglich.
    Nach der Anmeldung wird der Benutzer auf die Startseite der App weitergeleitet.
    Auf dieser Startseite werden die wichtigsten Funktionen (Kurse, Ideen, Zertifikate und Challenges) und die Kurse, in die der Benutzer eingeschrieben ist, angezeigt.
    In der oberen rechten Ecke ist ein Zahnrad-Symbol, über das der Benutzer auf die Einstellungen zugreifen kann.
    In der App lässt sich lediglich der Vor- und Nachname ändern, alle anderen Einstellungen sind auf der Webseite zu ändern.
    Viele der Funktionen der Webseite sind aus Gründen der Kompatilität in der App als sogenannte WebViews eingebunden, d.~h.~die CodeUp Webseite wird in die App eingebunden.
    Dies gilt für Zertifikate, Challenges, CodeUp Kids, Code Snippets, Editor (v1, v2, v3), Forum und Aufgaben.
    Flussdiagramme verfügen zwar über eine native Ansicht, allerdings nur für die Erstellung und Auswahl von Flussdiagrammen.
    Für die Bearbeitung wird ebenfalls ein WebView verwendet.
    Nativ-Implementiert sind Kurse, Ideen, das Discovery, Blog-Beiträge, Einstellungen und die Kontakt-Seite.
    Diese Hybrid-Lösung wurde bewusste gewählt, um die App möglichst schnell und einfach entwickeln zu können.
    Allerdings hat die Hybrid-Lösung auch für den Benutzer Vorteile, denn dieser bekommt immer die aktuellen Funktionen von CodeUp, ohne dass er ein Update der App durchführen muss.
    Die einzigen aktuell nicht verfügbaren Funktionen sind die Verwaltung von Organisationen, die Erstellung von Kurse und das Admin-Panel.
    \section{Plattformkompatibilität}
    Die CodeUp-Mobile App ist aktuell nur für iOS verfügbar.
    Dies liegt an verschärften Veröffentlichungsrichtlinien im Google Play Store, die eine Veröffentlichung der CodeUp-App erschweren.
    Rein technisch kann die App aber auch auf Android-Geräten genutzt werden, da React Native plattformunabhängig ist.
    Allerdings ist die App nicht für Android optimiert und kann daher auf Android-Geräten nicht die volle Leistungsfähigkeit entfalten.
    Eine Veröffentlichung im Google Play Store ist für die Zukunft geplant.
    \section{Fazit}
    Die CodeUp-Mobile App ist eine sinnvolle Ergänzung zur Webseite und bietet den Nutzern ein noch besseres Erlebnis.
    Die App ist einfach zu bedienen und bietet alle wichtigen Funktionen von CodeUp.
    Allerdings ist die App aktuell nur für iOS verfügbar, eine Veröffentlichung im Google Play Store ist für die Zukunft geplant.
    Damit die App gut in der Schule genutzt werden kann, müssen allerdings noch einige Funktionen implementiert werden, wie z.~B. die Verwaltung von Organisationen.
    Auch einige Fehler mit den WebViews müssen in Zukunft noch behoben werden, da diese manchmal nicht korrekt angezeigt werden.
    Insgesamt ist die CodeUp-Mobile App aber ein großer Erfolg und wird von den Nutzern gut angenommen.
\end{document}
