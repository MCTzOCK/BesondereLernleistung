\documentclass[main.tex]{subfiles}

\begin{document}
    CodeUp ist eine Plattform, die Anfängern und besonders jungen Schülern einen einfachen Einstieg in die Programmierung bietet.
    In dieser Arbeit wird die zeitliche Entwicklung der Plattform, die technischen Hintergründe und die didaktischen Überlegungen erläutert.
    Dabei werden auch auf die Probleme und Herausforderungen berücksichtigt, die während der Entwicklung aufgetreten sind.
    Grundsätzlich ist CodeUp in drei Teile gegliedert:
    \begin{itemize}
        \item Die Webseite, die den Kern des Projekts darstellt und das Hauptwerkzeug für die Nutzer ist.
        \item Das Spiel, das als Anreiz für die Schüler dient, sich mit dem Thema zu beschäftigen.
        \item Die Infrastruktur, die dafür sorgt, dass alle Komponenten miteinander kommunizieren können.
    \end{itemize}
    \section{Motivation}
    Die Motivation für CodeUp kam bei der Suche nach einem geeigneten Thema für ein Jugend forscht Projekt im Jahr 2023.
    Bei der Recherche stellte sich heraus, dass es in Deutschland einen erheblichen Fachkräftemangel im Bereich IT gibt\footnote{\bscite{bitkom-it-mangel}}.
    Dadurch wurde mir schnell klar, dass es wichtig ist, schon frühzeitig Interesse für das Thema zu wecken.
    Da ich selbst schon früh angefangen habe zu programmieren, war es naheliegend, ein Projekt zu entwickeln, das auch anderen Schülern den Einstieg erleichtert.
    \section{Zielsetzung}
    Das Ziel von CodeUp ist es, Schülern einen einfachen Einstieg in die Programmierung zu ermöglichen, hierbei wurde folgenden Punkten besondere Beachtung geschenkt:
    \begin{itemize}
        \item Benutzerfreundlichkeit: Die Plattform soll so einfach wie möglich zu bedienen sein.
        \item Funktionalität: Es sollen alle wichtigen Funktionen vorhanden sein, um die Schüler optimal zu unterstützen.
        Dabei soll möglichst jeder Aspekt der Programmierung abgedeckt werden.
        \item Zugriff: Die Plattform soll für alle Schüler zugänglich sein, unabhängig von ihrem Standort, ihrer finanziellen Situation oder ihren technischen Voraussetzungen.
    \end{itemize}
    \section{Aufbau der Arbeit}
    Die Arbeit ist in vier Oberkapitel gegliedert: Webseite, Mobile-App, Spiel und Infrastruktur.
    Dabei behandelt jedes Kapitel einen Teilbereich von CodeUp die zusammen die vollständige Plattform ergeben.
    Die Webseite ist dabei der Hauptteil des Projekts und bildet das Herzstück von CodeUp.
    Es wird ein besonderer Fokus auf die technischen Hintergründe gelegt, da diese für das Verständnis der Plattform entscheidend sind.
    Die Mobile-App ist eine Erweiterung der Webseite und bietet zusätzliche Funktionen, die auf mobilen Geräten besser genutzt werden können.
    Das Spiel dient als Anreiz für die Schüler, sich mit dem Thema zu beschäftigen und bietet eine spielerische Möglichkeit, die Programmierung zu erlernen.
    Im letzten Kapitel wird die Infrastruktur von CodeUp erläutert, die zum einen die Kommunikation zwischen den verschiedenen Komponenten ermöglicht und zum anderen das Bereitstellen der Softwarekomponenten ermöglicht.
    Das letzte Kaputel beinhaltet ebenfalls die Integration von DevOps-Praktiken, die die Entwicklung und Bereitstellung von CodeUp erleichtern.
\end{document}
