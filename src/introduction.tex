\documentclass[main.tex]{subfiles}

\begin{document}
    CodeUp ist eine Plattform, die Anfängern und besonders jungen Schülern einen einfachen Einstieg in die Programmierung bietet.
    In dieser Arbeit wird die zeitliche Entwicklung der Plattform, die technischen Hintergründe und die didaktischen Überlegungen erläutert.
    Dabei werden auch auf die Probleme und Herausforderungen berücksichtigt, die während der Entwicklung aufgetreten sind.
    Grundsätzlich ist CodeUp in drei Teile gegliedert:
    \begin{itemize}
        \item Die Webseite, die den Kern des Projekts darstellt und das Hauptwerkzeug für die Nutzer ist.
        \item Das Spiel, das als Anreiz für die Schüler dient, sich mit dem Thema zu beschäftigen.
        \item Die Infrastruktur, die dafür sorgt, dass alle Komponenten miteinander kommunizieren können.
    \end{itemize}
    \section{Motivation}
    Die Motivation für CodeUp kam bei der Suche nach einem geeigneten Thema für ein Jugend forscht Projekt im Jahr 2023.
    Bei der Recherche stellte sich heraus, dass es in Deutschland\footnote{\bscite{bitkom-it-mangel}}
\end{document}
