\documentclass[main.tex]{subfiles}

\begin{document}
    \section{InCode}\label{sec:incodedoc}
    InCode ist eine einfache Programmiersprache, die auf grammatikalisch möglichst korrekten Sätzen basiert.
    Entwickelt wurde InCode für Jugend forscht 2022 von Ben Siebert und Lukas Birke in der Kategorie Mathematik/Informatik.
    Beim Wettbewerb erreichte das Projekt den 1.~Platz auf Landesebene in der Junioren-Sparte des Wettbewerbs.
    Dabei wurde darauf geachtet, dass die Sprache möglichst einfach zu erlernen ist und auch für Anfänger geeignet ist.
    Die Grundidee ist, dass jeder ohne große Vorkenntnisse eigene kleine Web-Anwendungen entwickeln kann.
    Die Dokumentation von InCode ist unter~\url{https://incode.codeup.space} zu finden und der originale InCode-Editor ist unter~\url{https://incode.ben-siebert.com} zu erreichen.
    InCode kann auch vollumfänglich in CodeUp unter Verwendung der IDE (v1) oder des Editors (v2) benutzt werden.
    Allerdings ist in CodeUp eine verbesserte Version von InCode integriert, sodass die Dokumentation von InCode nicht vollständig auf CodeUp zutrifft.
    Es ist die unterliegende Sprache für die Entwicklung mit CodeUp-Kids.
    Der originale Quell-Code von InCode ist unter~\url{https://github.com/InCodeDevs/InCode} zu finden und die ursprüngliche Projektarbeit unter~\url{https://download.ben-siebert.com/projects/incode/InCode_essay_2022.pdf}.
    \section{React Native}\label{sec:react-native}
\end{document}
