\documentclass[12pt,a4paper]{report}

\usepackage[T1]{fontenc}
\usepackage[utf8]{inputenc}
\usepackage{charter}
\usepackage{titlesec}
\usepackage{tikz}
\usepackage{pgfplots}
\usepackage{amsmath}
\usepackage{amssymb}
\usepackage{subfiles}
\usepackage{ngerman}
\usepackage[left=2cm,right=2cm,top=2cm,bottom=2cm]{geometry}
\usepackage{graphicx}
\usepackage{tabularx}
\usepackage{hyperref}
\usepackage{biblatex}
\addbibresource{sources.bib}


\setcounter{tocdepth}{4}
\renewcommand{\baselinestretch}{1.5}\normalsize
\titleformat{\chapter}{\normalfont\LARGE}{\thechapter.}{15pt}{\Large}\titlespacing*{\chapter}{0pt}{*0}{10pt}
\newcommand{\bscite}[1]{\citeauthor{#1}: \citetitle{#1} (Stand: \citedate{#1}) \citeurl{#1}}

\begin{document}
    \title{Besondere Lernleistung}
    \author{Ben Siebert \\ Im Mühlenwinkel 14 \\ 45525 Hattingen}
    \date{
        \begin{tabularx}{14cm}{XX}
            \textbf{Titel}: & CodeUp - Einfacher Einstieg in die Programmierung \\
            \textbf{Betreuer}: & Herr A. Werner \\
            \textbf{Schule}: & Städtisches Gymnasium im Schulzentrum Holthausen \\
            \textbf{Leichtfach}: & Informatik \\
            \textbf{Abiturjahrgang}: & 2025 \\
            \textbf{Version}: & \today \\
        \end{tabularx}
    }
    \maketitle

    \tableofcontents

    \chapter{Einleitung}
    \subfile{introduction.tex}
    \chapter{Webseite}
    \chapter{Spiel}
    \chapter{Infrastruktur}
    \chapter{Literaturverzeichnis}
    \printbibliography[heading=none]
    \chapter{Anhang}
\end{document}
