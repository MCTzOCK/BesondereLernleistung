\documentclass[12pt,a4paper]{report}

\usepackage[T1]{fontenc}
\usepackage[utf8]{inputenc}
\usepackage{charter}
\usepackage{titlesec}
\usepackage{needspace}
\usepackage{etoolbox}
\usepackage{listings}
\usepackage{tikz}
\usepackage{pgfplots}
\usepackage{amsmath}
\usepackage{amssymb}
\usepackage{subfiles}
\usepackage{ngerman}
\usepackage[left=2cm,right=3cm,top=2cm,bottom=2cm]{geometry}
\usepackage{graphicx}
\usepackage{tabularx}
\usepackage{hyperref}
\usepackage{xcolor}
\usepackage[backend=biber]{biblatex}
\addbibresource{sources.bib}

\definecolor{cyellow}{HTML}{F7DE1F}
\definecolor{cred}{HTML}{E03131}

\lstdefinelanguage{javascript}{
    keywords={typeof, new, true, false, catch, function, return, null, catch, switch, var, if, in, while, do, else, case, break},
    keywordstyle=\color{blue}\bfseries,
    ndkeywords={class, export, boolean, throw, implements, import, this},
    ndkeywordstyle=\color{darkgray}\bfseries,
    identifierstyle=\color{black},
    sensitive=false,
    comment=[l]{//},
    morecomment=[s]{/*}{*/},
    commentstyle=\color{purple}\ttfamily,
    stringstyle=\color{red}\ttfamily,
    morestring=[b]',
    morestring=[b]"
}



\setcounter{tocdepth}{5}
\setcounter{secnumdepth}{5}
\renewcommand{\baselinestretch}{1.5}\normalsize
\titleformat{\chapter}{\normalfont\LARGE}{\thechapter.}{15pt}{\Large}\titlespacing*{\chapter}{0pt}{*0}{10pt}
\newcommand{\bscite}[1]{\citeauthor{#1}: \citetitle{#1} (Stand: \citedate{#1}) \citeurl{#1}}

\preto{\subsubsection}{\Needspace{5\baselineskip}} % Adjust space if needed


\begin{document}
    \title{Besondere Lernleistung}
    \author{Ben Siebert \\ Im Mühlenwinkel 14 \\ 45525 Hattingen}
    \date{
        \begin{tabularx}{14cm}{XX}
            \textbf{Titel}: & CodeUp - Einfacher Einstieg in die Programmierung \\
            \textbf{Betreuer}: & Herr A. Werner \\
            \textbf{Schule}: & Städtisches Gymnasium im Schulzentrum Holthausen \\
            \textbf{Leichtfach}: & Informatik \\
            \textbf{Abiturjahrgang}: & 2025 \\
            \textbf{Version}: & \today \\
        \end{tabularx}
    }
    \maketitle

    \tableofcontents

    \chapter{Einleitung}
    \subfile{introduction.tex}
    \chapter{Webseite}
    \subfile{website.tex}
    \chapter{Mobile-App}
    \chapter{Spiel}
    \chapter{Infrastruktur}
    \subfile{infrastructure.tex}
    \chapter{Literaturverzeichnis}
    \printbibliography[heading=none]
    \chapter{Anhang}
\end{document}
